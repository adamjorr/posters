\documentclass{beamer}
\mode<presentation>
\usepackage{tikz}
\usepackage{graphicx}
\usepackage{tabu}
\usepackage{xcolor}
\usetikzlibrary{positioning}
\usetikzlibrary{calc}
\usetikzlibrary{backgrounds}
\usefonttheme{professionalfonts}
\usetheme{Orr}
\usepackage[orientation=portrait, size=custom, width=111.76, height=111.76, scale=1.4]{beamerposter}
\usepackage{fontawesome}
\usepackage{booktabs}
\usepackage[citestyle=authoryear]{biblatex}
\bibliography{poster}{}

\graphicspath{{./figures/}{./figures/generated/}{./figures/static/}}

\title{KBBQ: A reference-free method for base quality score recalibration}
\titlegraphic{
	\includegraphics[width=.95\linewidth]{biodesign_logo.pdf}
	}
\date{3/22/19}
\author{Adam J Orr \inst{1,2} \faTwitter @AdamJOrr, Reed A Cartwright \inst{1,2} \faTwitter @MinionLab}
\institute{\inst{1} School of Life Sciences, Arizona State University \\
		   \inst{2} Biodesign Institute, Arizona State University}
\email{\faEnvelopeO \space ajorr1@asu.edu}
\website{\faLink \space   cartwrig.ht/lab/}

% \setlength{\topsep}{0pt}
% \setlength{\partopsep}{0pt}
\begin{document}

\begin{frame}{}
\begin{columns}[T]

%%%% Left side %%%%

\begin{column}[T]{.5\linewidth}

%%%%%%%%%%%%%%%%%%%

% KBBQ: A REFERENCE-FREE METHOD FOR BASE QUALITY SCORE
% RECALIBRATION
% Adam J Orr1,2, Reed A’ Cartwright1,2
% 1
% Arizona State University, School of Life Sciences, Tempe, AZ, 2
% Arizona

% State University, Biodesign Institute, Tempe, AZ
% Next-generation DNA sequencing is commonly used to detect genetic
% variation in many organisms. Quality scores are associated with the
% sequencing data that predict the accuracy of the base call in each read and
% statistical models for detecting variants utilize these scores to determine
% which bases are reliable and which are not. Despite their importance, base
% quality scores are often poorly calibrated and underestimate the probability
% of error, causing overconfidence in the accuracy of the resulting variant
% calls. Base quality score recalibration is a process to increase the accuracy
% of quality scores, which produces more accurate inferences. However, most
% methods of base quality score recalibration require a reference genome and
% database of variable sites, making them difficult to use and less accurate in
% non-model organisms where the reference may be of poor quality and there
% is no a priori knowledge of which sites are potentially variable. We present
% KBBQ, a reference-free method for base quality score recalibration that
% performs similarly to previous methods but can be used without a reference.
% KBBQ is MIT-licensed and available at www.github.com/adamjorr/kbbq


\begin{block}{Introduction}

\begin{columns}
\column{.72\linewidth}
\begin{itemize}
\item Illumina sequencing reads contain errors
\item Errors make mutation detection difficult
\item Quality scores represent $P(error)$ on a phred scale.
\begin{displaymath}
P(error) = 10^{\frac{-Q}{10}}
\end{displaymath}
\begin{displaymath}
Q = -10\log_{10}{P(error)}
\end{displaymath}
% \item Though quality scores are usually in the range \textbf{0 - 40}, to reduce the size of data files they are sometimes binned %into \textbf{7} bins 
\item Quality scores are sometimes binned to reduce file size
\end{itemize}
\column{.25\linewidth}
\begin{tabular}{l l r} \toprule
\input{qtable1.tex}
\end{tabular}
\end{columns}
\end{block}

\begin{block}{\textcolor{black}{Methods:} Base Quality Score Recalibration - BQSR}
Base Quality Score Recalibration is a technique to improve calibration of the original quality scores.
However, \textbf{BQSR normally requires a reference genome and a database of variable sites}.
Our reference-free method uses a method similar to the \texttt{lighter} (\cite{lighter}) error corrector to find erroneous bases rather than comparing the sequence to a reference. You can use your favorite error corrector and use those corrections as input to \texttt{kbbq}.

\begin{columns}
\column{.6\linewidth}
\begin{figure}
\begin{center}

\textbf{GATK BQSR - Reference Required}

\begin{tikzpicture}[framed, align=center, sqnode/.style={rectangle,draw=black!60,fill=black!5,very thick,minimum size=1cm}, node distance = 4 cm]
	\node[sqnode] (sequencing) {Sequence};
	\node[sqnode] (alignment) [right = of sequencing] {Alignment};
	\node[sqnode] (ignorevar) [right = of alignment] {Ignore Variable Sites};
	\node[sqnode] (lookupref) [below = 2cm of alignment] {Assume Reference Mismatches Are Errors};
	\node[sqnode] (lnmodel) [below = 2cm of lookupref] {Predict Error Rate with Linear Model};
	\draw[line width=3mm,->] (sequencing) -- (alignment);
	\draw[line width=3mm,->] (alignment) -- (ignorevar);
	\draw[line width=3mm,->] (ignorevar) -- (lookupref);
	\draw[line width=3mm,->] (lookupref) -- (lnmodel);
\end{tikzpicture}
\end{center}
\end{figure}

\column{.4\linewidth}
\begin{figure}
\begin{center}

\textbf{Reference Free BQSR}

\begin{tikzpicture}[framed, align=center, sqnode/.style={rectangle,draw=black!60,fill=black!5,very thick,minimum size=1cm}, node distance = 4 cm]
	\node[sqnode] (sequencing) {Sequence};
	\node[sqnode] (kmer) [below = 2cm of sequencing] {Kmer-based Method to Find Errors};
	\node[sqnode] (lnmodel) [below = 2cm of kmer] {Predict Error Rate with Linear Model};
	\draw[line width=3mm,->] (sequencing) -- (kmer);
	\draw[line width=3mm,->] (kmer) -- (lnmodel);
\end{tikzpicture}

\end{center}
\end{figure}
\end{columns}
\end{block}

\begin{block}{\textcolor{black}{Methods:} Hierarchical Linear Model}

\begin{columns}
\column{.55\linewidth}

The recalibration uses a hierarchical linear model to predict the true probability of error given a set of covariates.
Each covariate causes the predicted quality score to shift up or down from the predicted score one level above it in the hierarchy.

\vfill

The relevant covariates are:
\begin{itemize}
	\item Read Group
	\item Original Assigned Quality Score
	\item Position in read and whether read is forward or reverse (Cycle)
	\item Base called and the prior base call (Context)
\end{itemize}

\column{.4\linewidth}

\begin{figure}
\begin{center}

\textbf{Hierarchical Linear Model}

\begin{tikzpicture}[align=center, sqnode/.style={rectangle,draw=black!60,fill=black!5,very thick,minimum size=1cm}, node distance = 4 cm]
	\node[sqnode,align=center,above,minimum size=16cm](rg) at (8,0){};
	% \draw[help lines] (0,0) grid (16,16);
	\node[](rglab) at (8,15){Read Group};
	\node[sqnode, align=center,above,minimum size=13cm,fill=gold!10](q) at (8,1){};
	\node[](qlab) at (8,12){Assigned Quality Score};
	\node[sqnode, minimum size = 5 cm,fill=maroon!10] at (5,6) {Cycle};
	\node[sqnode, minimum size = 5 cm,fill=maroon!10] at (11,6) {Context};
\end{tikzpicture}
\end{center}
\end{figure}
\end{columns}
\end{block}
%\textbf{Algorithm:} %For each base, record its read group, assigned quality score, cycle, context, and whether or not it was an error.
%Record the error rate of each read group, then subset each read group by the assigned quality score and calculate the error rate for each subset.
%Further subset these by 1) cycle and 2) context, and calculate the error rate for each subset.
% \begin{block}{\textcolor{black}{Methods:} Algorithm}
% Subset each base into groups according its covariates.
% Use the mean base quality of the entire dataset as an initial prior. For each level in the hierarchy, the prior is normally distributed around the 
% result of the previous level. Given this prior, find the maximum a posteriori (MAP) error rate and record the difference between the prior and the MAP.
% The recalibrated quality is the sum of these differences.

% % \vskip 2.2ex

% \end{block}


\begin{block}{\textcolor{black}{Methods:} Evaluation}
\begin{itemize}
	% \item CHM1 and CHM13 are human complete hydatidiform mole cell lines.
	% \item The moles are formed when a sperm combines with an egg containing no nucleus.
	% \item The sperm then undergoes mitosis to generate a completely homozygous cell mass.
	\item DNA from CHM1 and CHM13 human hydatidiform mole cell lines was mixed to generate a synthetic diploid dataset for benchmarking (\cite{li_syndip}).
	\item We analyzed Illumina reads aligned on Chromosome 1 and compared recalibrated quality scores generated using our method and the standard GATK method.
\end{itemize}
\end{block}


\begin{block}{Next Steps}
\begin{itemize}
	\item Test performance of GATK BQSR with varying levels of false positives.
	% \item Simulate reads from a distant genome and compare performance with GATK BQSR.
	\item Linear model improvements
	\item Programming optimizations.
\end{itemize}
\end{block}


\begin{block}{Software}

\faicon{github} \url{https://github.com/adamjorr/kbbq}

\end{block}

\begin{block}{Acknowledgements}

% \begin{center}

\begin{columns}
%\column{.4\linewidth}
%This work is supported by grants NIH R01-HG007178 and NSF DBI-1356548.
\column{.27\linewidth}
\includegraphics[width=\linewidth]{lab_logo.pdf}
\column{.27\linewidth}
\includegraphics[width=\linewidth]{sols_logo.pdf}
\column{.3\linewidth}
\includegraphics[width=\linewidth]{biodesign_logo.pdf}
\end{columns}

% \end{center}

\end{block}

% %%% Right Side %%%%
\end{column}
\begin{column}[T]{.5\linewidth}

%%%%%%%%%%%%%%%%%%%%%%%

\begin{block}{\textcolor{black}{Results:} GATK BQSR is vulnerable to false negatives in the known sites input.}
\begin{center}
\includegraphics[width=.986\textwidth]{fnr.pdf}
\end{center}
KBBQ doesn't require \textit{a priori} knowledge of variable sites, so is unaffected by misspecification of known sites.
\end{block}

\begin{block}{\textcolor{black}{Results:} Similar performance, no reference or variants required}
\begin{center}
\includegraphics[width=.986\textwidth]{comparison.pdf}
\end{center}
Both methods improve calibration and resolution, while KBBQ has relaxed input requirements.
% The numbers in the legend are the Brier scores of each calibration method. The Brier score penalizes incorrect predictions,
% so a perfect prediction method will have a Brier score of 0. GATK BQSR improves the Brier score by $0.00047$ while our method
% improves the Brier score by $0.00043$. The peak at Q = 2 is because 2 is a reserved Illumina quality control mark and
% represents an invalid base.
\end{block}


\end{column}
\end{columns}
\end{frame}

\end{document}